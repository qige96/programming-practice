\documentclass[12pt]{extarticle}
\usepackage[utf8]{inputenc}
\usepackage{cite}

\title{Preliminary Report on Norm Replacement}
\author{Ruiqi Zhu}
\date{May 2020}

\begin{document}

\maketitle
\setlength{\parskip}{0.5em}

\begin{abstract}
This preliminary report serves as a project proposal which describes a research project on the study of norm emergence. 

\end{abstract}

\section{Introduction}
In particular, this project focus on the phenomenon of norm replacement, i.e. the process that a new norm emerges and replaces the old one. 

This report firstly introduce the proposed project in a big picture, including the fundamental concepts, as well as motivations. Next, the background section generally looks through the field of norm emergence, and provides a critical literature review of previous research on norms and related works. Then, the methodology section presents the main techniques and experimental settings.  Lastly, a brief project schedule will be given, demonstrating the key stages of the research project and a respective timeline. 

\subsection{Norm, and Norm Emergence}
\textbf{What is norm?} The term \textit{Norm} is indeed a vague word in our daily use, so lots of researchers give their own definitions on norm in accordance with their understanding and research context. However, norm is usually considered as common or expected behaviour in society. 

Axelrod, who made the groundbreaking work on the realm of norm emergence with agent-based modelling in 1986\cite{axelrod1986evolutionary}, gave his definition of norm as followed: "A norm exists in a given social setting to the extent that individuals usually act in a certain way and are often punished when seen not to be acting in this way."

Further, Tuomela categorised norm in a more meticulous way: r-norms (rule norms), s-norms (social norms), m-norms (moral norms) and p-norms (prudential norms)\cite{tuomela_1996}. Rule norms are essentially laws and regulations that are imposed to follow. Social norms essentially customs and conventions applied to a large group of individuals. Moral norms are regarded as one's principle of bahaviour derived from individual’s conscience, and prudential norms are designed or concluded practices with careful rational consideration. In multiagent systems (MAS), r-norms and s-norms are most studied, and the norms come from emergence are s-norms, similar to Axelrod's definition.

In this report, without specification, the term norm refers to s-norms, or rather, the norm defined by Axelrod.

\textbf{What are the applications of norm?} Norm is vital in our daily life for it can maintain human behaviour patterns that are acceptable to the majority of the population, which helps our society run efficiently. So does it in multiagent systems. An agent is an autonomous virtual or real entities such as software programs, robots, or self-driving car. In a broader context, people and other organisms could also be viewed as agents. Multiagent system consist of a number of agents that can interact with one another. Agents might assemble to solve a big problem collectively, like routers directing messages in the Internet, or they might just happened to get together, like self-driving cars on the road. Yet for them to successfully interact, they need to be capable of cooperating, coordinating, and negotiating. Norm can serve these purposes for MAS, just as it for human society.

\textbf{What is norm emergence?}
Emergence indicates the phenomenon that micro level individuals interact with one another to bring about macro level patterns in a bottom-up manner. Scientists study norm emergence because they want to have a theory accounting how norms come into being, spread and change. Also, in engineering fields, scientists wish to exploit emergent methods, bringing norms into multiagent systems rather than designing them in a prescriptive way, which is considered inefficient and unrobust.

\subsection{Motivation}
In 2009, a literature review written by Savarimuthu\cite{Savarimuthu2011} stated that "None of the works that are based on simulation address how a norm is made obsolete or how a new norm replace an old one." After that, there are some empirical study on the change of norms, but the majority of that happened in traditional social science field\cite{Gest2013} \cite{Sandholtz2019}. However, still there aren't much work focusing on the change or the replacement of norms, especially in engineering aspects. 

Norm replacement could be a subject of great significance. Agent based simulation is now a powerful tool for investigating complex systems. One aim of agent based simulation is to explore intervention. For example, by simulation we can identify the factors that influence the system most, and then apply intervention in the real world to make changes happen. 

Intervention is essentially a norm replacement process: use one norm to replace the old one. Thus, we ought to know more about the norm replacement process, such as phases division, and factors that determine the speed and convergence ratio of the replacement process, etc.


\section{Background}
Axelrod made his grounding work on norm emergence.

Axelrod's work is so inspiring that many of the following scientists replicate and extend his work to explore this realm. Savarimuthu's extend his work.

Luck's work investigated the influence of network topology.

Three papers on norm life cycle.

Many other literature about norm change, norm decay, norm replacement, in social science subjects.

Agent based modelling (ABM), or agent based simulation, applies the the concept of agent and multiagent system to the the task of system modelling. To build an agent based model, one should specify three elements: agents attributes and behaviours, interactions rules, and the environment. Then we could see how the agents move and interact and thus cause the whole system to evolve.

Compared with physical modelling, ABM is more flexible, efficient, and cheap (both in time and financial cost). Compared with mathematical modelling, ABM provides more readable and human-friendly visual representation. Most importantly, agent based modelling is natural to depict the micro entities, and the relationship among them in a  complex system, which makes the model concrete, and closer to the real world phenomenon. Thus, agent based modelling is particularly powerful in representing complex system, and revealing the hidden patterns, and \textbf{emergent properties} of the system. Currently, ABM has been applied to modelling ecosystem, economy, human society, physics particle swarm, and many other fields.

There had been research\cite{Yang2019} examining 17 ABM applications in health behavior and behavior intervention. The author focused on two perspectives: the mechanism of behavior and behavior change, and ABMs' use for behavior intervention. The review reassured the important insights obtained from the use of ABM, and that the research interests of ABM in behaviour intervention is still growing.


\section{Aims and Objectives}
In this proposed project, our aim is to gain understandings of norm replacement process in support of decision making or other practice, such as launching a system transformation in an effective way.

An ideal plan to study norm replacement could be to build a universal agent-based model for norm game, and then apply changes(interventions) to the model during simulation, so as to observe what changes could trigger norm replacement, and how the replacement would occur. However, it is not easy to work out that universal model. So far, each agent-based models is specific to certain problem domain, and thus each model has its own parameters and mechanism. But still, there is one obvious and general factor that worth studying. which is \textbf{the pace of applying changes} to those models. For instance, in Savarimuthu's work\cite{Savarimuthu2011}, there is a model parameter called "punishment cost", which stands for the negative payoff for each time an agent implements punishment. When a norm was formed, to study the effect of pace of intervention, I could change the punishment cost from 1 to 0.01 suddenly, or change it from 1 to 0.01 progressively by 0.01 every 100 iterations.

Therefore, practical objectives could be: 

\begin{itemize}

\item Modifying model parameters in different pace to see whether norm replacement would occur and how is the process like.
\end{itemize}


\section{Methodology}
In this project, we are going to adopt agent based modelling to simulate norm emergence in a multiagent society, and explore the norm replacement process.

\subsection{Experiment Design}
We use Savarimuthu's work as our starting point. The experiment setup is a typical agent-based model. In a grid society, a group of \textit{N} agents move randomly. Agents have a private state \textit{Score}, initialised to be 100. Every iteration an agent could choose to Liter or not Liter. If an agent liters, it gets a payoff 0.5, else -0.5. What's more, Every Litering action would harm every agent in this society, with a negative payoff 1/N. Under some conditions non-litering agents could decide to punish those litering agents, but the punisher would need to afford the punishment cost \textit{Pcost}, a negative payoff either. Agents also have a private parameter called \textit{autonomy}, initialised randomly from 0 to 9, indicating the number of punishments required by an agent to move from L to NL. Another parameter of this model is minimum Survival score \textit{minScore}. When agents' own score go below \textit{minScore}, they flip their strategy(from L to NL, or vice versa). These are the basic settings of the simulation experiment.

Starting simulation, agents move and each iteration there are some agents meet with each other, and can observe the other's action. Assume that initially 50 of 100 agents liter and the rest don't liter, and a certain proportion of non-litering agents are punishers that would incur punishment to the litering agents they meet. \textit{Pcost} could be 0.01, 0.1, 1, or 10. We could run teh simulation for a very long run and see if a norm emerge.

After a norm was formed, we could change some parameters to see if a new norm would rise and the old one would fade out. For example, if a NL norm was formed (all agents become NL agents), we change the \textit{Pcost} on the fly. To study the effect of changing pace, we could change the \textit{Pcost} suddenly from 0.01 tp 10 within one iteration, or progressively increase the value of \textit{Pcost} by 0.01 every 10 iterations. To adapt for this intervention, we need to adjust the rule of litering: even if an agent is of type NL, it still have a low probilility, 0.001 for example, to liter.  


\section{Schedule}
Considering that this research is relatively simple to conduct, and my exam period ends on June 2nd, I am planning to start experiment from June 3rd, and finish all experiments and analysis in July (almost 2 months). Then the whole August is for thesis writing.


\bibliographystyle{plain}
\bibliography{M335}

\end{document}
