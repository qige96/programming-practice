\documentclass[12pt]{extarticle}
\usepackage[utf8]{inputenc}
\usepackage{cite}

\title{Preliminary Report on Study of Norm Replacement}
\author{Ruiqi Zhu}
\date{June 2020}

\begin{document}

\maketitle
\setlength{\parskip}{0.5em}

\begin{abstract}
Norm is regarded as a powerful tool in understanding society, as well as in regulating agent behaviours of multiagent system engineering. There has been plenty of effort put on study of norm emergence, but when it comes to norm replacement, we seems to know little. This preliminary report serves as a project proposal which describes a research project on the study of norm replacement. In our proposed project, we would like to study on the basis of an existing metanorm game model, about the whether withdrawing metanorm mechanism would trigger a norm replacement process. Also, we are going to test if the pace of parameter modification (punishment cost) would have influence on norm replacement.
\end{abstract}

\section{Introduction}
Norm emergence refers to the phenomena that micro level individuals interact with one another, and together give rise to macro level system properties. The emergence is spontaneous, and usually the norms are expredictable, that's what attract research interests. In particular, this project focus on norm replacement, i.e. the process that a new norm emerges and replaces the old one. 

This report firstly introduce the proposed project in a big picture, including the fundamental concepts, as well as motivations. Next, the background section generally looks through the field of norm emergence, and provides a critical literature review of previous research on norms and related works. Then, our aims and objectives would be clearly stated. The Following methodology section presents the main techniques and experimental settings.  Lastly, a brief project schedule will be given, demonstrating the key stages of the research project and a respective timeline. 

\subsection{Norm, and Norm Emergence}
\textbf{What is norm?} The term \textit{Norm} is indeed a vague term in our daily use, so lots of researchers give their own definitions on norm in accordance with their understanding and research context. However, norm is usually considered as common or expected behaviour in society. 

Axelrod, who made the groundbreaking work on the realm of norm emergence with agent based modelling in 1986\cite{axelrod1986evolutionary}, gave his definition of norm as followed: "A norm exists in a given social setting to the extent that individuals usually act in a certain way and are often punished when seen not to be acting in this way."

Further, Tuomela categorised norm in a more meticulous way: r-norms (rule norms), s-norms (social norms), m-norms (moral norms) and p-norms (prudential norms)\cite{tuomela_1996}. Rule norms are essentially laws and regulations that are imposed to follow. Social norms essentially customs and conventions applied to a large group of individuals. Moral norms are regarded as one's principle of bahaviour derived from individual’s conscience, and prudential norms are designed or concluded practices with careful rational consideration. In multiagent systems (MAS), r-norms and s-norms are most studied, and the norms come from emergence are s-norms, similar to Axelrod's definition.

In this report, without specification, the term norm refers to s-norms, or rather, the norm defined by Axelrod.

\textbf{What are the applications of norm?} Norm is vital in our daily life for it can maintain human behaviour patterns that are acceptable to the majority of the population, which helps our society run efficiently. So does it in multiagent systems. An agent is an autonomous virtual or real entities such as software programs, robots, or self-driving car. In a broader context, people and other organisms could also be viewed as agents. Multiagent system consist of a number of agents that can interact with one another. Agents might assemble to solve a big problem collectively, like routers directing messages in the Internet, or they might just happened to get together, like self-driving cars on the road. Yet for them to successfully interact, they need to be capable of cooperating, coordinating, and negotiating. Norm can serve these purposes for MAS, just as it for human society.

\textbf{What is norm emergence?}
Emergence indicates the phenomenon that micro level individuals interact with one another to bring about macro level patterns in a bottom-up manner. Scientists study norm emergence because they want to have a theory accounting how norms come into being, spread and change. Also, in engineering fields, scientists wish to exploit emergent methods, bringing norms into multiagent systems rather than designing them in a prescriptive way, which is considered inefficient and unrobust.

\subsection{Motivation}
Axelrod stated his research motivation of research on norm emergence as "...we need a good theory of norms. Such a theory should help explain three things: how norms arise, how norms are maintained, and how one norm displaces another."\cite{axelrod1986evolutionary} However, his  experiment only explored the former two questions, without explicit effort on how one norm displaces another. Under what conditions could a norm be replaced by another? How is the dynamics of norm replacement? What are the factors that determine the speed and convergence ratio of the replacement process? How does the replacement process could be divided into phases? These interesting questions remained to be answered.

Norm replacement could be a subject of great significance. For instance, intervention is essentially a norm replacement process: use one norm to replace the old one. Scientists had been thinking about modelling the real world systems with agent based simulation, and identify factors that influence the system most, then apply intervention in the real world to make changes happen. Therefore, to make sure that the intervention decisions are effective, we ought to know more about the norm replacement process, such as phases division, and factors that determine the speed and convergence ratio of the replacement process, etc.

\section{Background}
% Norm emergence
Among all the literature about norm emergence, Axelrod's paper\cite{axelrod1986evolutionary} may be considered as the groundbreaking work to successfully formalise norm emergence in the context of agent based modelling, and drew meaningful conclusions. Axelrod began to think with famous game theoretic problem, prisoner's dilemma. He hoped to find out a method for a group of people in lack communication to form a norm of cooperation, instead of defection. His solution was an evolutionary game with metanorm mechanism. He implemented experiments using agent based modelling, and shown that if non-defecting agents could punish defecting agents in a low cost, making the overall payoff of defection lower than cooperation, then a norm of cooperation could be established and held.

Axelrod's work is so inspiring that many of the following scientists replicated, extended, and analysed his work to explore this realm. Purvis, Purvis, and Savarimuthu's experiments\cite{Savarimuthu2008} tested Axelrod's notion in a larger scale virtual online society and proved that social norms could be successfully developed to prevent small defections where the cost of enforcement is low. They also tested the idea that common knowledge were helpful to form and stablise a norm. Mahmoud and other researchers from King's College London and University of Warwick conducted a research on extending Axelrod's framework\cite{mahmoud2017establishing}, and demonstrated that the use of iterative game and metanorm could be effective in establishing norms in broader configurations, including lattices space and small world network.

% Norm replacement
Successive researchers conducted their research on various topics, like norm emergence in different spatial or network structures\cite{burke2006emergence}\cite{10.1145/3127498}\cite{10.1007/978-981-10-2564-8_14}\cite{mahmoud2017establishing}, or other mechanisms other than evolutionary computation that could give birth to norms\cite{sen2007emergence}\cite{villatoro2011social}. Yet, in 2009, a literature review written by Savarimuthu\cite{Savarimuthu2011} stated that "None of the works that are based on simulation address how a norm is made obsolete or how a new norm replace an old one." After that, there were some empirical study on the change of norms, but the majority of that happened in traditional social science field\cite{Gest2013} \cite{Sandholtz2019}. However, still there aren't much work focusing on the change or the replacement of norms, especially in engineering aspects.

% Agent Based Modelling
One widely used technique in norm emergence, and broadly speaking, computational social science research, is called agent based modelling. 

Agent based modelling (ABM), or agent based simulation, applies the the concept of agent and multiagent system to the the task of system modelling. To build an agent based model, one should specify three elements: agents attributes and behaviours, interactions rules, and the environment. Then we could see how the agents move and interact and thus cause the whole system to evolve.\cite{klugl2012agent}

Compared with physical modelling, ABM is more flexible, efficient, and cheap (both in time and financial cost). Compared with mathematical modelling (equation-based modelling), ABM provides more readable and human-friendly visual representation. Most importantly, agent based modelling is natural to depict the micro entities, and the relationship among them in a complex system, which makes the model concrete, and closer to the real world phenomena. Thus, agent based modelling is particularly powerful in representing complex system, and revealing the hidden patterns, and \textbf{emergent properties} of the system. Currently, ABM has been applied to modelling cellular biophysics\cite{bayrak2014agent}, economy\cite{arthur2006out}, human society\cite{an2012modeling}, ecosystem\cite{heckbert2010agent}, and many other fields.

Agent based simulation is now also being applied in exploring intervention, i.e. norm replacement. There had been research\cite{Yang2019} examining 17 ABM applications in health behavior and behavior intervention. The author focused on two perspectives: the mechanism of behavior and behavior change, and ABMs' use for behavior intervention. The review reassured the important insights obtained from the use of ABM, and that the research interests of ABM in behaviour intervention is still growing.


\section{Aims and Objectives}
In this proposed project, our aim is to gain understandings of norm replacement process in support of decision making or other practice, such as launching a system transformation in an effective way. In particular, we would like explore norm replacement on the basis of Savarimuthu's model\cite{Savarimuthu2008} about the following questions. 

\begin{itemize}
\item What would happen if the metanorm mechanism were withdrawn? Will a new norm replace the old one?
\item What would happen if the punishment cost be changed radically, like increasing suddenly from 0.01 to 10?
\item What would happen if the punishment cost be changed progressively, like increasing by 0.01 every 100 iterations, from 0.01 to 10? 
\end{itemize}

\section{Methodology}
In this project, we are going to adopt agent based modelling to simulate norm emergence in a multiagent society, and explore the norm replacement process.

We use Savarimuthu's work\cite{Savarimuthu2008} as our starting point. The experiment setup is a typical agent based model. In a grid society, a group of \textit{N} agents move randomly. Agents have a private state \textit{Score}, initialised to be 100. Every iteration an agent could choose to Liter or not Liter. If an agent liters, it gets a payoff 0.5, else -0.5. What's more, every Liter action would harm every agent in this society, with a negative payoff 1/N. Non-litering agents could decide to punish those litering agents if they meet during the random movement, but the punisher would need to afford the punishment cost \textit{Pcost}, a negative payoff either. Agents also have a private parameter called \textit{autonomy}, initialised randomly from 0 to 9, indicating the number of punishments required by an agent to move from L to NL. Another parameter of this model is minimum Survival score \textit{minScore}. When agents' own score go below \textit{minScore}, they flip their strategy(from L to NL, or vice versa). These are the basic settings of the simulation experiment.

Starting simulation, agents move and each iteration there are some agents meet with each other, and can observe the other's action. Assume that initially 50 of 100 agents liter and the rest don't liter, and a certain proportion of non-litering agents are punishers that would incur punishment to the litering agents they meet. \textit{Pcost} could be 0.01, 0.1, 1, or 10. We could run ten simulation for a very long run and see if a norm emerge.

After a norm was formed, we could apply interventions to the model on the fly, so as to see if a new norm would arise and the old one would fade out. Supposed that a NL norm was formed (all agents become NL agents):

Firstly, we could remove the sanction mechanism from the model by banning punishment from non-litering agents to litering agents, and observe if the norm shift from NL to L, and how much time spent.

To study the effect of changing pace, we could change the \textit{Pcost} suddenly from 0.01 tp 10 within one iteration, and observe if the norm shitf from NL to L, and how much time spent.

Then, we could change the \textit{Pcost} progressively increase the value of \textit{Pcost} by 0.01 every 10 iterations, and observe if the norm shift from NL to L, and how much time spent.

Of course, to adapt for these interventions, we need to adjust the model, such as the behaviour of litering: even if an agent is of type NL, it still have a low probability, 0.001 for example, to liter.

\section{Schedule}
Considering that this research is relatively simple to conduct, here is a brief timeline for this project. 

\begin{tabular}{l|c}
\hline
Date      &  Milestone \\\hline
Aug 21st  & submitting final report  \\
Aug 1st   & starting to write final report  \\
Jul 15th  & explore question 3  \\
Jul 1st   & explore question 2  \\
Jun 9th   & implement model, explore question 1 \\
\hline
\end{tabular}

\bibliographystyle{plain}
\bibliography{M335}

\end{document}


