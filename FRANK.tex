\documentclass[12pt]{extarticle}
\usepackage[utf8]{inputenc}
\usepackage{cite}

\title{Enhancing the FRANK System for Answering Non-numeric Query}
\author{Ruiqi Zhu\\
ruiqi_zhu@outlook.com\\
}
\date{June 2020}

\begin{document}

\maketitle
\setlength{\parskip}{0.5em}

\begin{abstract}
As the quantity and quality of available knowledge base on the web are now increasing rapidly, we could use them to do better with query answering, not only by means of information retrieval, but also inference. This proposal described the basic mechanism of the FRANK query answering system, whose outstanding characteristic is to infer out new knowledge from the existing information in the knowledge base, including making predictions. FRANK is currently focused on numerical queries, only some limited kinds of qualitative queries. It would be a natural improvement to extend the system to answer more types of non-numeric queries. The proposal details what type of non-numeric queries is to be investigated, and presents a solution: superclass and subclass decomposition and related inference methods, together with an acceptable uncertainty measure, weighted credibility. At last, more types of non-numeric queries would be discussed. 
\end{abstract}

\section{Introduction}
Semantic Web\cite{berners2001semantic} and other semi-structured data are of unimaginable value. A great number of applications, especially knowledge base question answering systems were built to exploit the data. However, most of these applications are not yet matured to be ready to use, and also, they basically use only look-up techniques, which means directly searching for a piece of data. To make the most use of Semantic Web, inference and other techniques that could refine new knowledge from existing information are necessary. 

\subsection{Background information}
A lot of attempts have been made to exploit linked data and infer out new knowledge, among which the RIF (Rich Inference Framework) system\cite{nuamah2016functional}, now evolved into FRANK (Functional Reasoning Acquire New Knowledge) system \cite{bundy2018automated} is remarkable. RIF can not only handle queries that require deductive and statistical inference over numerical facts, but also appropriately measure uncertainty for each answer using  error bar, a metric developed from confidence interval. Moreover, when dealing with a query, the system could decompose the query into sub-queries, and eventually aggregate all interim results and produce a finally one. The FRANK system introduced a common representation called *alist* to universally represent knowledge from all sources, and the query, And also now explain its inference. Also, FRANK provides a simple natural language interface, that can translate user questions into *alist* format queries, and also now explain its inference. 

Despite that the RIF and FRANK system made such a huge breakthrough, there are still much work to do. One of the most value-added job is to extend the system capability to handle inference with non-numeric queries. Originally, RIF and FRANK system targeted at numeric queries, so that some inference methods were even only applicable for inferring numbers (e.g. MAX,  MIN, AVG, MODE, REGRESS). To process non-numeric queries, the system might need some new inference methods and even new inference strategies, as well as corresponding uncertainty measurements.

\subsection{Brief summary of existing literature}
QALD \cite{unger2014question} stands for Question Answering over Linked Data. This is probably the largest competition for question answering using knowledge base. But most system implementation in QALD literatures do only information retrieval, without inferring new knowledge from the old.  Watson \cite{ferrucci2010building} is a famous query answering system that defeat human champions on quiz game  Jeopardy! The architecture of Watson, DeepQA, incorporates various QA related techniques, from machine learning regression to statistical and deductive inference. and provides much inspirations, and it uses not only knowledge base but also texts, relational databases, and other structured or semi-structured data. There are also many others query answering solutions for example graph embedding based question answering over knowledge graph \cite{huang2019knowledge}. Yet these methods pay much attention to statistical modelling, merely little of them put effort on other kinds of  approaches, such as deductive reasoning.


\section{Problem}
The key research problem is to investigate how the FRANK system could be extended to answer more types of non-numeric queries.

\subsection{Numeric query}
To clarify what is the non-numeric query are we aiming at, let us first look at how is numeric queries look like, and how the FRANK system deals with this kind of query. A typical query is "*Which country will have the largest population in Africa in 2021?*" To answer this query, the FRANK system firstly applies a lookup operation to its knowledge base, with an attempt to find a piece of information that could directly response to the query without further inference. If this attempt fails, inference process starts. 

\begin{itemize}
\item  At first geospatial decomposition operation would be applied to expand the term "country in Africa", and then again lookup the knowledge base for population in 2021 in each country in Africa. 

\item  Once the population data could not be found, temporal decomposition operation would be applied to find out all available population data in all years of each country in Africa. After that, regression operation would be applied to these population data, and simple prediction for population of each country in Africa in 2021 could be made.

\item  Since (inferred) population data of all African countries are now available, a simple max operation could be applied and the country with the largest population in Africa in 2021 is returned.
\end{itemize}

\begin{figure}
    \centering
    \includegraphics[scale=0.7]{images/numeric-search-tree.png}
    \caption{FRANK's Search Tree for the query "Which country will have the largest population in Africa in 2021?" Taken from the original paper\cite{bundy2018automated}}
    \label{Fig:1}
\end{figure}

Although the query is looking for a discrete value, a particular country name, **all the operation involved in the inference process are all arithmetic ones (except decompositions)**. They are virtually all about finding the maximal, minimal, sum, average or regression on real-valued numbers. Yet operations for handling non-numeric projection variables, such as boolean or discrete values are missing, together with the corresponding uncertainty estimation.

\subsection{Non-numeric query}
According to numeric queries above, non-numeric queries should be regarded as those  whose inference process involve handling non-numeric (intermediate or final) variables. 

In fact, inference with respect to non-numeric queries have been studied for a long time.  Traditional deductive and inductive reasoning concluded by  Aristotle are typical non-numeric inference. Classical propositional logic studying true value of a combination of statements could be seen as a kind of non-numeric query, whose inference process is all about boolean values. Moreover, inference on knowledge base, like the Semantic Web, is also a kind of non-numeric query, which could be characterised as mining for new relationships\cite{dalwadi2012semantic} among entities and classes. 

By intuition, a typical non-numeric query required inference is "Will Socrates die?" To infer for the answer (if not stored in the knowledge base), we follow such a process: All persons will die and Socrates is a person, so Socrates will die. This is a deductive style inference. One more query "Were all ancient Greek philosophers male? " follows an inductive style inference: first find out all ancient Greek philosophers, and then check the gender for all of them.


\section{Method}
\subsection{The hypothesis and the objectives}
\textit{A combination of information retrieval and deductive, arithmetic and statistical reasoning could be used to answer numeric and non-numeric, or hybrid queries, with a reliable uncertainty estimate.}

\subsection{Superclass and Subclass Decomposition}
It is always expected that a magical methodology can universally solve all problems, at least in one given domain. Unfortunately such a methodology does not always exist, and finding it is always difficult. Therefore, we have to propose specific strategies for specific problems.

FRANK constructs AND/OR search tree to conduct its inference for numeric queries, so should it for non-numeric queries. Of course it might need new decomposition operation, aggregation operation, and corresponding selection strategy. Here  some new operations are proposed:

\begin{itemize}
\item Superclass Decomposition. This decomposition find out all direct superclass and apply VALUE, ANY or ALL operation, aggregating boolean values.
\item Subclass Decomposition. This decomposition find out all direct subclass and apply VALUE, ANY or ALL operation, aggregating boolean values.
\end{itemize}

With these new operations, it is possible to conduct more traditional deductive and inductive reasoning and answer boolean-valued queries. 

\begin{figure}
    \centering
    \includegraphics[scale=0.7]{images/2.png}
    \caption{FRANK's Search Tree for the query "Will Socrates die?"}
    \label{Fig:2}
\end{figure}

\subsection{Uncertainty measure}
FRANK system assigns uncertainty value for every answers to a query. For numerical answers, an error bar is calculated using CoV. But for non-numerical answers, CoV might not be suitable since CoVs are associated with Gaussian distributions, which are fundamentally numeric. A new uncertainty measure is probably  needed, and here comes a new challenge: interleaving all kinds of uncertainty measures during inference. 

Uncertainty can come from many aspects. It may from the knowledge sources, of which some knowledge bases are badly created and some records are outdated. It may from inference strategy selection. Considering these, a **weighted credibility** might be an acceptable solution. In advanced, aspects of credibility are pre-defined, with corresponding weights. Credibility could be pre-assigned, such as for knowledge base, or computed during inference, such as  recency, or even retrieved from knowledge records —— some knowledge base, like Wikidata\cite{vrandevcic2014wikidata}, provide references or data quality clues for data items. 

Suppose a new uncertainty measure is introduced, here comes a new problem: how to interleave two uncertainty measures during the propagation phase of inference? Will there be such a case that after decomposition we get a numerical subquery and a non-numerical subquery, and the former one is assigned with an error bar yet the latter one is assigned  with another uncertainty measure, like a weighted credibility? If so, how to combine the two measures into one final measure? These remain to be explored further.


\section{More Non-numeric Queries}
There could be much more potential types of non-numeric queries.

Another non-numeric query required inference may be "Who are Mary's grandchildren?"  There might not be direct answers in the knowledge base stating who are Mary's grandchildren. But there should be records stating who are Mary's children, and the children of those children. Then if we can reason out that "children's children" means "grandchildren", we could find out Mary's grandchildren.

A complicated non-numeric query required inference as well as commonsense reasoning might be "which country does Mary live now?" There might not be direct answers in the knowledge base stating where does Mary live now, but perhaps we could find out that Mary currently works in UK. Commonsense tells that a person is very likely to live and work in the same place, though not necessary. Thus, we might infer out that Mary live in UK now, with some uncertainty. 

A much more complicated query might be "which country did Mary live for the longest time?" In this case, we may not only use commonsense reasoning, but also geospatial and temporal decomposition and various arithmetic operations in the FRANK system. In short, this might require hybrid inference.

For more non-numeric queries, more decomposition and aggregation operations might be needed. Since inference is a broadly studied field, we could easily borrow ideas from neighbouring branches, such as automated theorem proving\cite{bibel2013automated}, or first-order logic reasoning\cite{fitting2012first}. Corresponding uncertainty, or credibility estimate could use a similar weighted credibility calculation.


\section{Conclusion}
This proposal aimed at extending the FRANK capability to answer non-numeric queries, specifically one type of non-numeric query that required traditional deductive and inductive reasoning, with new decomposition strategy and inference methods, and that will be the major challenge. Moreover, new uncertainty measure for this kind of queries is likely to be  invented, and how to interleave that with error bars for numerical answers will be a new problem. Superclass and subclass decomposition are mainly taxonomy based, they could help handle a large kind of queries about class inference, which will contribute to the FRANK system. But the exact  performance is to be examined after further investigation and real implementation. Also, more kinds of non-numeric queries is to be explored. There are still much work to do.



\bibliographystyle{plain}
\bibliography{M335}

\end{document}


